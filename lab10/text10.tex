\documentclass{article}
\usepackage{blindtext}
\usepackage{geometry}
 \geometry{
 a4paper,
 total={170mm,257mm},
 left=30mm,
 right=30mm,
 top=26mm,
 bottom=35mm,
 }
\usepackage{graphicx}
\graphicspath{ {images/} }
\pagenumbering{arabic}
\usepackage[utf8]{inputenc}
\usepackage[english,russian]{babel}
\usepackage{mathtools}
\usepackage{amssymb}
\setcounter{page}{158}
\DeclareMathOperator{\sign}{sign}
\newcommand\tab[1][0,5cm]{\hspace*{#1}}
\linespread{1,5}


\begin{document}
\begin{figure}[h]
\center{\includegraphics[width=1\linewidth]{графиклатех}}
\end{figure}
{\fontsize{15,8pt}{18pt}\selectfont
\tabИз общего определения графика функции (см. п. 1.2*) следует, что график функции $y = f(x)$ ($x$ и $y$ --- числа, $x \in X$) представляет собой множество точек 
($x, f(x)$), $x \in X$, на кооорднатной плоскости переменных $x$ и $y$.}

{\fontsize{15,8pt}{18pt}\selectfont
\tabТак, график функции (5.1) имеет вид, изображенный на рисунке 18, график функции $\sign x$ (см. формулы (5.2)) --- на рисунке 19, а график функции $y = 1 + \sqrt{\lg \cos 2\pi x}$ состоит из отдельных точек, соответствующих целым значениям аргумента $x = 0$, $\pm1$, $\pm2$, $\ldots$, так как при остальных значениях аргумента выражение под знаком радикала принимает отрицательные значения (рис. $20$).}

{\fontsize{15,8pt}{18pt}\selectfont
\tabМножество точек \{$(x, y)$: $x \in X, y \geqslant f(x)$\} называется \textit{надграфиком} данной функции $f$, а множество $\{(x, y)$: $x \in X$, $y \leqslant f(x)\}$ --- ее \textit{подграфиком}.}

{\fontsize{15,8pt}{18pt}\selectfont
\tabГрафическое изображение функции также может служить для задания функциональной зависимости. Правда, это задание будет приближенно потому, что измерениие отрезков практически можно производить лишь с определенной степенью точности. Примерами графческого задания функций, встречающимися на практике, могут служить, например, показания осциллографа.}

{\fontsize{15,8pt}{18pt}\selectfont
\tabФункцию можно задать  с п о м о щ ь ю  т а б л и ц, т. е. для некоторых значений переменной $x$ указать соответствующие значения переменной $y$. Данные таблиц могут быть получены как непосредственно из опыта, так и с помощью тех или иных математических расчетов. Примерами такого задания функций являются логарифмические таблицы тригонометрических функций. Само собой разумеется, что функция, заданная с помощью таблицы, определена}

\begin{center}
   \line(1,0){70}
\end{center}
\newpage
\end{document}