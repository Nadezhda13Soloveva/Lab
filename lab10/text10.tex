\documentclass[12pt, legalpaper]{article}
\usepackage{graphicx}
\graphicspath{ {images/} }
\pagenumbering{arabic}
\usepackage[utf8]{inputenc}
\usepackage[english,russian]{babel}
\usepackage{mathtools}
\usepackage{amssymb}
\setcounter{page}{158}
\DeclareMathOperator{\sign}{sign}

\begin{figure}[h]
\center{\includegraphics[width=1\linewidth]{графиклатех}}
\end{figure}

\begin{document}
\large
Из общего определения графика функции (см. п. 1.2*) следует, что график функции $y = f(x)$ ($x$ и $y$ --- числа, $x \in X$) представляет собой множество точек 
($x, f(x)$), $x \in X$, на кооорднатной плоскости переменных $x$ и $y$.

Так, график функции (5.1) имеет вид, изображенный на рисунке 18, график функции $\sign x$ (см. формулы (5.2)) --- на рисунке 19, а график функции $y = 1 + \sqrt{\lg \cos 2\pi x}$ состоит из отдельных точек, соответствующих целым значениям аргумента $x = 0$, $\pm1$, $\pm2$, $\ldots$, так как при остальных значениях аргумента выражение под знаком радикала принимает отрицательные значения (рис. $20$).

Множество точек \{$(x, y)$: $x \in X, y \geqslant f(x)$\} называется \textit{надграфиком} данной функции $f$, а множество $\{(x, y)$: $x \in X$, $y \leqslant f(x)\}$ --- ее \textit{подграфиком}.

Графическое изображение функции также может служить для задания функциональной зависимости. Правда, это задание будет приближенно потому, что измерениие отрезков практически можно производить лишь с определенной степенью точности. Примерами графческого задания функций, встречающимися на практике, могут служить, например, показания осциллографа.

Функцию можно задать  с п о м о щ ь ю  т а б л и ц, т. е. для некоторых значений переменной $x$ указать соответствующие значения переменной $y$. Данные таблиц могут быть получены как непосредственно из опыта, так и с помощью тех или иных математических расчетов. Примерами такого задания функций являются логарифмические таблицы тригонометрических функций. Само собой разумеется, 
\newline
\newline
\newline
\newline
\begin{center}
   \line(1,0){70}
\end{center}

\newpage
что функция, заданная с помощью таблицы, определена на конечном множестве точек.


Наконец, припроведении числовых расчетов на компьютерах функции задаются с помощью программ для их вычисления при нужных значениях аргумента или требуемые значения функции в готовом виде закладываются тем или иным способом в память компьютера.

Рассмотрим более подробно некоторые специальные аналитические способы задания функции.

Н е я в н ы е  ф у н к ц и и. Пусть дано уравнение вида
\begin{equation}\tag{5.3}
F(x, y) = 0,    
\end{equation}
т. е. задана функция $F(x, y)$ двух действительных переменных вида $x$ и $y$, и рассматриваются только такие пары $x, y$ (если они существуют), для которых выполняется условие (5.3).

Пусть существует такое множество $X$, что для каждого $x_0 \in X$ существует по крайней мере одно число $y$, удовлетворяющее уравнению $F(x_0, y_0) = 0$. Обозначим одно из таких чисел через $y_0$ и поставим его в соответствие числу $x_0 \in X$. В результате получим функцию $f$, определенную на множестве $X$ и такую, что $F(x_0, f(x_0)) = 0$ для всех $x_0 \in X$. В этом случае говорят, что функция $f$ задается неявно уравнением ($5.3$). Одно и то же уравнение ($5.3$) задает, вообще говоря, не одну, а некоторое множество функций.

Функции, неявно задаваемые уравнениями вида (5.3), незываеются \textit{неявными функциями} в отличие от функций, задаваемых формулой, разрешенной относительно переменной $y$, т. е. формулой вила $y = f(x)$.

Термин <<неявная функция>> отражает не характер функциональной зависимости, а лишь способ ее задания. Одна и та же функция может быть задана как явно, так и неявно. Например, функции $f_1(x) = \sqrt{1 - x^2}$ и $f_2(x) = -\sqrt{1 - x^2}$ могут быть заданы также и неявным образом с помощью уравнения $x^2 + y^2 - 1 = 0$ в том смысле, что они входят в совокупность функций, задаваемых этим уравнерием.

С л о ж н ы е  ф у н к ц и и. Напомним, что если заданы функции $y = f(x)$ и $z = F(y)$, причем область задания функции F содержит область значений функции f, то каждому $x$ из области определения функции f естественным образом со-
\newline
\newline
\begin{center}
   \line(1,0){70}
\end{center}
\end{document}
